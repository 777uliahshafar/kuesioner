\documentclass[11pt]{article}

% personalise page geometry, e.g., margin,
% for your printer and needs
\usepackage[
  a4paper,
  left = 20mm,
  top = 20mm,
  bottom = 15mm,
]{geometry}

% switch off page numbering
\pagenumbering{gobble}

% use for generating tables
\usepackage{csvsimple}
\csvstyle{GSscale}{
  tabular=r|c|c|c|c|c|l,
  table head =
      & 1 & 2 & 3 & 4 & 5 &\\\hline,
  late after line = \\\hline,
  head=false
}
\csvstyle{RSscale}{
  tabular=r|c|c|c|c|c|c|c|, 
  table head = 
      & 1 & 2 & 3 & 4 & 5 & 6 & 7\\\hline, 
  late after line = \\\hline,
  head=false
}

% for circle box
\usepackage{wasysym}

% indents in itemizes
\usepackage{enumitem}

% switch to sans serif
\renewcommand{\familydefault}{\sfdefault}

% factor for table padding
\renewcommand{\arraystretch}{1.6}

% spacing for itemizes
\usepackage{setspace}
\setlength{\parindent}{0cm}
% \doublespacing
\setlist{leftmargin=1.5cm,itemsep=.5\baselineskip,before=\vspace{.5\baselineskip},after=\vspace{\baselineskip}}

% define the robot's name
\def \robot {Care-O-Bot}

% multiple optional arguments in newcommand, loops
\usepackage{xparse}
\usepackage{pgffor}

% handy for open questions
\newcommand{\openquestion}[1]{#1\\[1em]%
\underline{\hspace{14cm}}\\[1.5em]%
\underline{\hspace{14cm}}\\[1.5em]%
\underline{\hspace{14cm}}\\[1.5em]}

\newcounter{scaleCounter}
\newcommand{\createLikertHeader}{


}

% handy for likert-style questions
\NewDocumentCommand{\likertquestion}{ O{Strongly disagree} O{Strongly agree} O{5} m }{%
\setcounter{scaleCounter}{0}
\def\ltablecfg{}
\loop\ifnum\thescaleCounter<#3
  \stepcounter{scaleCounter}
  \edef\ltablecfg{%
    \ltablecfg
    c|
  }%
\repeat

\setcounter{scaleCounter}{0}
\def\lheader{}
\loop\ifnum\thescaleCounter<#3
  \stepcounter{scaleCounter}
  \edef\lheader{%
    \lheader
    \arabic{scaleCounter} &
  }%
\repeat

\setcounter{scaleCounter}{0}
\def\lbullets{}
\loop\ifnum\thescaleCounter<#3
  \stepcounter{scaleCounter}
  \edef\lbullets{%
    \lbullets
    $\ocircle$ &
  }%
\repeat

\begin{center}%
#4\\[1em]%
\begin{tabular}{r|\ltablecfg l}%
 & \lheader \\%
\hline%
#1 & \lbullets #2\\
\hline%
\end{tabular}%
\\[2em]
\end{center}%
}


\setrobot{Example Robot}

\begin{document}

\section*{Kuesioner}
\subsection*{Data Pribadi}

\begin{itemize}
\item[Partisipan\#:]\underline{\hspace{2cm}}
\end{itemize}

Jenis Kelamin:
\begin{multicols}{2}
\begin{itemize}
\item[$\ocircle$] Laki-laki \item [$\ocircle$] Perempuan
\end{itemize}
\end{multicols}

Ras:
\begin{multicols}{3}
\begin{itemize}
\item[$\ocircle$] Bugis
\item[$\ocircle$] Makassar
\item[$\ocircle$] Toraja
\item[$\ocircle$] Duri
\item[$\ocircle$] Lainnya \underline{\hspace{2cm}}
\end{itemize}
\end{multicols}

\begin{itemize}
\item[Tahun lahir:] \underline{\hspace{2cm}} Tahun
\end{itemize}

Pekerjaan:
\begin{multicols}{3}
\begin{itemize}
\item[$\ocircle$] Karyawan
\item[$\ocircle$] Wiraswasta
\item[$\ocircle$] Belum Bekerja
\item[$\ocircle$] Pelajar
\item[$\ocircle$] Pensiun
\end{itemize}
\end{multicols}


\subsection*{Pertanyaan 1}
Jika uang, waktu, dan halangan lainnya bukan pertimbangan, ruang mana yang saudara lebih cenderungi untuk dikunjungi? Silakan nilai dengan skala poin 100  dimana ``1" artinya ``kurang dicenderungi" dan ``100" artinya "sangat dicenderungi".

    \begin{tabular}{p{.38\textwidth} c}
    \textbf{Ruang} & \textbf{Nilai kecenderungan}\\
    \end{tabular}
\vspace{10pt}

\score{Ruang A}
\score{Ruang B}

\subsection*{Pertanyaan 2}
\begin{itemize}
    \item Mengapa saudara menyukai ruang tersebut? Deskripsikan dengan detail apa yang menarik menurut saudara tentang atribut ruang itu?

    \item Bagaimana perasaan saudara saat memasuki ruang tersebut?
\end{itemize}
\begin{itemize}
    \item Seberapa sering saudara menggunakan ruang tersebut?
\end{itemize}

\begin{description}
    \item [1] = Belum pernah dengar
    \item [2] = Belum pernah kesana
    \item [3] = Pernah kesana sekali
    \item [4] = Pernah kesana lebih sekali
\end{description}
% Table preferensi ruang
\begin{center}
  \csvreader[ShortScale]{MDRAS/MDRAS-items.csv}{}%
  {\raggedleft \csvcoli & $\ocircle$ & $\ocircle$ & $\ocircle$ & $\ocircle$}%
\end{center}

\subsection*{Pertanyaan 3}
Silakan nilai (centang) pendapat yang paling mewakili pendapat saudara dengan skala poin untuk setiap pernyataan. Dimana ``1" artinya tidak setuju ``5" artinya setuju. \\


Gunakan baris tambahan untuk komentar-komentar.

\begin{enumerate}[leftmargin=1em]
    \item \fourlikert[][]{Jalan setapak membantu menuntun pengunjung ke ruang publik.}
    \openquestion{\,}

    \item \fourlikert[][]{Lahan parkir yang banyak adalah sangat baik untuk setiap orang.}
    \openquestion{\,}

    \item \fourlikert[][]{Tempat yang terbuka memudahkan ruang publik untuk dikenali.}
    \openquestion{\,}

    \item \fourlikert[][]{Ruang publik adalah tempat yang terakses dan inklusif secara sosial dan harus digunakan oleh setiap orang tanpa harus bayar.}
    \openquestion{\,}

%----------------------------------------------------------------------------------------

    \item \fourlikert[][]{Saya kurang enak saat orang lain melihat apa yang saya lakukan.}
    \openquestion{\,}

    \item \fourlikert[][]{Pembatas atau jarak tempat duduk berkontribusi terhadap ruang pribadi (\textit{personal space}) setiap orang}
    \openquestion{\,}

    \item \fourlikert[][]{Vandalisme, sampah, grafiti memberikan kesan negatif ruang publik}
    \openquestion{\,}

    \item \fourlikert[][]{Lampu jalan atau taman dapat memberikan kesempatan untuk berwisata malam}
    \openquestion{\,}

%----------------------------------------------------------------------------------------

    \item \fourlikert[][]{Dapat diterima untuk melepas vegetasi yang ada demi menambahkan pemandangan yang menyorot keindahan alam.}
    \openquestion{\,}


    \item \fourlikert[][]{Warna dan material lantai memberikan gaya pada ruang publik.}
    \openquestion{\,}

    \item \fourlikert[][]{Saya lebih mungkin mengunjungi ruang publik yang bersih dan terpelihara.}
    \openquestion{\,}


%----------------------------------------------------------------------------------------

    \item \fourlikert[][]{Ruang komunal pada ruang publik diinginkan dan membantu interaksi sosial.}
    \openquestion{\,}

    \item \fourlikert[][]{Beberapa aktifitas seperti olahraga, diizinkan di ruang publik (seperti berenang, skateboard \& sepeda).}
    \openquestion{\,}

    \item \fourlikert[][]{Area alami di ruang publik memberikan jalan keluar dari tekanan hidup.}
    \openquestion{\,}

    \item \fourlikert[][]{Ruang gerak yang lebih banyak sangat baik untuk setiap orang.}
    \openquestion{\,}
\end{enumerate}


\begin{comment}
\newpage
\textbf{Kognitif}
\begin{enumerate}
    \item Area alami di kota memberikan jalan keluar dari tekanan kehidupan kota.
    \item Taman kota dan jalan setapak dapat memberikan kesempatan untuk belajar tentang alam.
    \item Bangunan atau struktur bersejarah yang dijaga di taman berkontribusi kepada pemahaman komunitas sejarawan.
\end{enumerate}

\textbf{Ekologi}
\begin{enumerate}
    \item Dapat diterima untuk membatasi akses ke semua tempat di taman untuk melindungi spesies atau proses alami yang penting.
    \item Taman perkotaan dan jalan setapak seharusnya menyorot keindahan penanaman naturalisasi, diadopsi ke kondisi lokal untuk menambah habitat spesies.
    \item Taman perkotaan dan ruang hijau adalah rumah untuk beragam tanaman dan hewan yang harus dilindungi.
    \item
\end{enumerate}

\textbf{Partisipatif}
\begin{enumerate}
    \item Jaringan tanaman dan jalan setapak yang terhubung dapat berfungsi sebagai rute transit alternatif ke jalan untuk bukan pengguna motor.
    \item Jalan setapak taman dan kota seringkali merupakan jalan pintas dan alternatif aman untuk jalan yang banyak diperdagangkan.
    \item Dalam sebuah taman, saya seringkali tertanggu oleh permainan riuh orang lain.
\end{enumerate}

\textbf{Adil}

\begin{enumerate}
    \item Taman dimaksudkan sebagai ruang "hanya anak"
    \item Taman inklusif secara sosial dan tempat aksessibel dan harus digunakan oleh siapa saja tanpa pengecualian atau biaya masuk
    \item Beberapa aktivitas seperti olahraga ekstrim dan alternatif, tidak diizinkan dilingkungan perkotaan lainnya, diizinkan di taman atau sepanjang setapak kota(seperti skateboard, sepeda bmx, skating)
\end{enumerate}

\textbf{Ekonomi}
\begin{enumerate}
    \item penggunaan komersial yang lebih banyak adalah sangat bagus untuk setiap orang
    \item festival dan produksi pada taman diinginkan dan membawa pendapatan tambahan kepada komunitas
    \item bisnis kecil, terletak di taman dapat memberikan pelayanan dan barang-barang kepada pengunjung taman.
\end{enumerate}

\textbf{Sensorik}
\begin{enumerate}
    \item seni publik, terletak di taman perkotaan dapat berkontribusi kepada identitas komunitas sekitar
    \item Dapat diterima melepas vegetasi yang ada untuk meningkatkan pemandangan yang signifikan, yang menonjolkan keindahan alam
    \item Saya lebih mungkin mengunjungi taman dan setapak perkotaaan di musim panas
\end{enumerate}







\section*{Others}
\subsection*{Pertanyaan 2}
%Follow this article Preferences of older people for environmental attributes of local parks- The use of choice‐based conjoint analysis
Silahkan tunjukkan seberapa pentingnya fitur ruang berikut ini ketika mencari tempat di tepi laut.

\likertquestion[Tidak Penting][Penting]{Akses}
\likertquestion[Tidak Penting][Penting]{Fasilitas}
\likertquestion[Tidak Penting][Penting]{Estetika}
\likertquestion[Tidak Penting][Penting]{Keamanan}
\likertquestion[Tidak Penting][Penting]{Pemeliharaan}


\subsection*{Pertanyaan 2}
Silahkan tunjukkan seberapa pentingnya fitur ruang berikut ini ketika mencari tempat di tepi laut.
%% Akses
\likertquestion[Tidak Penting][Sangat Penting]{Lebar sebuah jalan.}
\likertquestion[Tidak Penting][Penting]{Kedekatan fasilitas.}
%% Fasilitas
\likertquestion[Tidak Penting][Penting]{Kehadiran fitur buatan (cth. bangku, lampu jalan).}
\likertquestion[Tidak Penting][Penting]{Kehadiran fitur hijau (cth. tumbuhan, pohon, taman).}
\likertquestion[Tidak Penting][Penting]{Kehadiran fitur biru (cth. kolam, danau, laut).}
%% Estetika
\likertquestion[Tidak Penting][Penting]{Kerapatan vegetasi (tumbuhan, pohon).}
\likertquestion[Tidak Penting][Penting]{Kehadiran tumbuhan (cth. pohon, taman).}
%% Keamanan
\likertquestion[Tidak Penting][Penting]{Ketertutupan sebuah ruang}
\likertquestion[Tidak Penting][Penting]{Keramaian sebuah ruang}
%% Pemeliharaan
\likertquestion[Tidak Penting][Penting]{Kondisi Rumput}
\likertquestion[Tidak Penting][Penting]{Vandalisme}
\likertquestion[Tidak Penting][Penting]{Sampah}

\pagebreak

\subsection*{Pertanyaan 3}
%Silahkan nilai kesan anda terhadap ruang di tepi laut berikut dalam hal "kemudahan akses dengan lebar jalan yang tersedia" dengan skala 7 tingkat dimana "1" mewakili "sangat sulit" dan "7" artinya "sangat mudah".

%Berdasarkan lebar jalan dan kedekatan fasilitas, bagaimana anda menilai ruang di kawasan tepi laut dalam hal "kemudahan atau kesulitanmu untuk mengakses sebuah ruang di kawasan tepi laut" pada skala 7 tingkat dimana "1" mewakili "sangat sulit" dan "7" artinya "sangat mudah".

Berdasarkan lebar jalan dan kedekatan fasilitas, silahkan nilai presepsi anda terhadap ruang-ruang di kawasan tepi laut dalam hal "kemudahan atau kesulitanmu untuk mengakses sebuah ruang di kawasan tepi laut" pada skala 7 tingkat dimana "1" mewakili "sangat sulit" dan "7" artinya "sangat mudah".

    \begin{tabular}{p{.48\textwidth} c}
    & \textbf{Sangat Sulit <-------> Sangat Mudah}
    \end{tabular}

\begin{center}
  \csvreader[LongScale]{MDRAS/MDRAS-items.csv}{}%
  {\raggedleft \csvcoli & $\ocircle$ & $\ocircle$ & $\ocircle$ & $\ocircle$ & $\ocircle$ & $\ocircle$ & $\ocircle$}%
\end{center}


\subsection*{Pertanyaan 4}
Silahkan nilai presepsi anda terhadap ruang-ruang di kawasan tepi laut dalam hal "Ketersediaan fitur buatan (cth. bangku, lampu jalan)" pada skala 7 tingkat dimana "1" mewakili "tidak ada" dan "7" artinya "sangat banyak".


    \begin{tabular}{p{.48\textwidth} c}
    & \textbf{Tidak ada <-------> Sangat Banyak}
    \end{tabular}

\begin{center}
  \csvreader[LongScale]{MDRAS/MDRAS-items.csv}{}%
  {\raggedleft \csvcoli & $\ocircle$ & $\ocircle$ & $\ocircle$ & $\ocircle$ & $\ocircle$ & $\ocircle$ & $\ocircle$}%
\end{center}


\subsection*{Pertanyaan 5}
Silahkan nilai presepsi anda terhadap ruang-ruang di kawasan tepi laut dalam hal "Ketersediaan fitur hijau (cth. pohon, taman)" pada skala 7 tingkat dimana "1" mewakili "tidak ada" dan "7" artinya "sangat banyak".


    \begin{tabular}{p{.48\textwidth} c}
    & \textbf{Tidak ada <-------> Sangat Banyak}
    \end{tabular}

\begin{center}
  \csvreader[LongScale]{MDRAS/MDRAS-items.csv}{}%
  {\raggedleft \csvcoli & $\ocircle$ & $\ocircle$ & $\ocircle$ & $\ocircle$ & $\ocircle$ & $\ocircle$ & $\ocircle$}%
\end{center}


\subsection*{Pertanyaan 6}
Silahkan nilai presepsi anda terhadap ruang-ruang di kawasan tepi laut dalam hal "Ketersediaan fitur hijau (cth. pohon, taman)" pada skala 7 tingkat dimana "1" mewakili "tidak ada" dan "7" artinya "sangat banyak".


    \begin{tabular}{p{.48\textwidth} c}
    & \textbf{Tidak ada <-------> Sangat Banyak}
    \end{tabular}

\begin{center}
  \csvreader[LongScale]{MDRAS/MDRAS-items.csv}{}%
  {\raggedleft \csvcoli & $\ocircle$ & $\ocircle$ & $\ocircle$ & $\ocircle$ & $\ocircle$ & $\ocircle$ & $\ocircle$}%
\end{center}

\subsection*{Pertanyaan 7}
Silahkan nilai presepsi anda terhadap ruang-ruang di kawasan tepi laut dalam hal "Keindahan estetika (termasuk kerapatan vegetasi dan kehadiran tumbuhan)" pada skala 7 tingkat dimana "1" mewakili "sangat polos" dan "7" artinya "sangat indah".


    \begin{tabular}{p{.48\textwidth} c}
    & \textbf{Sangat Polos <-------> Sangat Indah}
    \end{tabular}

\begin{center}
  \csvreader[LongScale]{MDRAS/MDRAS-items.csv}{}%
  {\raggedleft \csvcoli & $\ocircle$ & $\ocircle$ & $\ocircle$ & $\ocircle$ & $\ocircle$ & $\ocircle$ & $\ocircle$}%
\end{center}


\subsection*{Pertanyaan 8}
Berdasarkan tingkat ketertutupan dan keramaian, silahkan nilai presepsi anda terhadap ruang-ruang di kawasan tepi laut dalam hal "rasa aman ketika berada di kawasan tepi laut Senggol" pada skala 7 tingkat dimana "1" mewakili "sangat takut" dan "7" artinya "sangat aman".


    \begin{tabular}{p{.48\textwidth} c}
    & \textbf{Sangat Takut <-------> Sangat Aman}
    \end{tabular}

\begin{center}
  \csvreader[LongScale]{MDRAS/MDRAS-items.csv}{}%
  {\raggedleft \csvcoli & $\ocircle$ & $\ocircle$ & $\ocircle$ & $\ocircle$ & $\ocircle$ & $\ocircle$ & $\ocircle$}%
\end{center}

\subsection*{Pertanyaan 9}

Silahkan nilai presepsi anda terhadap ruang-ruang di kawasan tepi laut dalam hal "pemeliharaan kawasan tepi laut (termasuk keadaan rumput, grafiti dan sampah)" pada skala 7 tingkat dimana "1" mewakili "sangat buruk" dan "7" artinya "sangat baik".


    \begin{tabular}{p{.48\textwidth} c}
    & \textbf{Sangat Takut <-------> Sangat Aman}
    \end{tabular}

\begin{center}
  \csvreader[LongScale]{MDRAS/MDRAS-items.csv}{}%
  {\raggedleft \csvcoli & $\ocircle$ & $\ocircle$ & $\ocircle$ & $\ocircle$ & $\ocircle$ & $\ocircle$ & $\ocircle$}%
\end{center}

\end{comment}
%-------------------------------------------------------------------------------------
% Reference
%-------------------------------------------------------------------------------------

\begin{comment}
% Comments
\subsection*{Comments}
\openquestion{Why did you do XYZ?}\\
\openquestion{Do you have any general impressions about \robot{} during your interaction with it?}\\
\openquestion{Do you have any questions you would like to ask us?}

\pagebreak

% RoSAS

\subsection*{Robot rating}
Using the scale provided, how closely do you associate the following attributes with \robot?%
\begin{center}
  \csvreader[RSscale]{RoSAS/RoSAS-items.csv}{}%
  {\csvcoli & $\ocircle$ & $\ocircle$ & $\ocircle$ & $\ocircle$ & $\ocircle$ & $\ocircle$ & $\ocircle$}%
\end{center}

\pagebreak



% GodSpeed
\subsection*{Robot rating}
Please rate your impression of \robot{} on these scales:%
\begin{center}
  \csvreader[GSscale]{GodSpeed/GodSpeed-items.csv}{}%
  {\csvcoli & $\ocircle$ & $\ocircle$ & $\ocircle$ & $\ocircle$ & $\ocircle$ & \csvcolii}%
\end{center}

\subsection*{Emotional state}
Please rate your own emotional state on these scales:
\begin{center}
  \csvreader[GSscale]{GodSpeed/GodSpeed-items-estate.csv}{}%
  {\csvcoli & $\ocircle$ & $\ocircle$ & $\ocircle$ & $\ocircle$ & $\ocircle$ & \csvcolii}%
\end{center}

\pagebreak


\end{comment}

% MDRAS

\end{document}
