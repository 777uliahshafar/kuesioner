\documentclass[11pt]{article}

% personalise page geometry, e.g., margin,
% for your printer and needs
\usepackage[
  a4paper,
  left = 20mm,
  top = 20mm,
  bottom = 15mm,
]{geometry}

% switch off page numbering
\pagenumbering{gobble}

% use for generating tables
\usepackage{csvsimple}
\csvstyle{GSscale}{
  tabular=r|c|c|c|c|c|l,
  table head =
      & 1 & 2 & 3 & 4 & 5 &\\\hline,
  late after line = \\\hline,
  head=false
}
\csvstyle{RSscale}{
  tabular=r|c|c|c|c|c|c|c|, 
  table head = 
      & 1 & 2 & 3 & 4 & 5 & 6 & 7\\\hline, 
  late after line = \\\hline,
  head=false
}

% for circle box
\usepackage{wasysym}

% indents in itemizes
\usepackage{enumitem}

% switch to sans serif
\renewcommand{\familydefault}{\sfdefault}

% factor for table padding
\renewcommand{\arraystretch}{1.6}

% spacing for itemizes
\usepackage{setspace}
\setlength{\parindent}{0cm}
% \doublespacing
\setlist{leftmargin=1.5cm,itemsep=.5\baselineskip,before=\vspace{.5\baselineskip},after=\vspace{\baselineskip}}

% define the robot's name
\def \robot {the robot}%
\newcommand{\setrobot}[1]{%
\def \robot {#1}%
}

% multiple optional arguments in newcommand, loops
\usepackage{xparse}
\usepackage{pgffor}

% handy for open questions
\newcommand{\openquestion}[1]{#1\\[1em]%
\underline{\hspace{14cm}}\\[1.5em]%
\underline{\hspace{14cm}}\\[1.5em]%
\underline{\hspace{14cm}}\\[1.5em]}

\newcounter{scaleCounter}
\newcommand{\createLikertHeader}{


}

% handy for likert-style questions
\NewDocumentCommand{\likertquestion}{ O{Strongly disagree} O{Strongly agree} O{5} m }{%
\setcounter{scaleCounter}{0}
\def\ltablecfg{}
\loop\ifnum\thescaleCounter<#3
  \stepcounter{scaleCounter}
  \edef\ltablecfg{%
    \ltablecfg
    c|
  }%
\repeat

\setcounter{scaleCounter}{0}
\def\lheader{}
\loop\ifnum\thescaleCounter<#3
  \stepcounter{scaleCounter}
  \edef\lheader{%
    \lheader
    \arabic{scaleCounter} &
  }%
\repeat

\setcounter{scaleCounter}{0}
\def\lbullets{}
\loop\ifnum\thescaleCounter<#3
  \stepcounter{scaleCounter}
  \edef\lbullets{%
    \lbullets
    $\ocircle$ &
  }%
\repeat

\begin{center}%
#4\\[1em]%
\begin{tabular}{r|\ltablecfg l}%
 & \lheader \\%
\hline%
#1 & \lbullets #2\\
\hline%
\end{tabular}%
\\[2em]
\end{center}%
}
